% Chapter 3 from the thesis template file

\chapter{METHODS AND PROCEDURES}

\section{Research Design}

The purpose of this descriptive study in the form of a survey is to determine the educational need for a carnivorous companion animal nutrition program at the community college level.  This study will report on the existing knowledge or perceived knowledge of veterinary technician, veterinary assistant, and animal science students at community colleges in Iowa in carnivorous companion animal nutrition. In addition to these students, practicing veterinary technicians, veterinary assistants, veterinarians, and other veterinary clinic employees were surveyed, and this study will report on those results, as well. This study also strives to reveal the future educational interest of the groups surveyed in the subject of carnivorous companion animal nutrition.  A survey sent to the community college students currently enrolled in the veterinary technician and assistant programs at Iowa community colleges would result in defining the need for a new nutrition program for carnivorous companion animals. Sending the same survey to those currently employed in veterinary clinics can result in supporting data for defining the need for a new community college program.
\par In utilizing a survey, there are certain validity issues to address, such as the appropriateness of the questions in the survey.  Questions unrelated to assessing existing knowledge of students or identifying areas of need or demographical questions would not need to be included \citep{ary}.  Reviews of the survey questions occurred to ensure that all questions are relevant to the research in some way.  Also, questions were worded and ordered in such a way to avoid respondents giving answers they wish were true or answers that they think the researchers want to see, which are both problems identified by \cite{ary}.  Ensuring anonymity of respondents will make sure that respondents are less likely to give “safe” answers if they feel their honest answers would have some sort of consequence \citep{ary}.



\section{Data Source}

The target population for this study is the population of community college students and future students that are enrolled or will enroll in the veterinary technician, veterinary assistant, and animal science programs of study in Iowa.  The accessible population for the duration of this study would be the current population of Iowa community college students in the previously mentioned fields of study.  This is the only portion of the population accessible at this point since the rest of the target population has not yet enrolled in the program of study.  However, community colleges have a privacy policy that disallows release of student e-mail addresses to anyone. In an effort to follow policy but still acquire student responses, course instructors and department heads were contacted at all Iowa community colleges with at least one of the relevant programs of study. Including all community colleges in Iowa with the three programs in question allows for some diversity in background and education to acquire more representative responses. All community college contacts were provided with an e-mail that could be forwarded to students and contained an anonymous survey link.
\par A secondary target population in this study is veterinary technicians, veterinary assistants, and even veterinarians currently practicing in the regions near the community colleges with the appropriate programs. To acquire survey responses from these professionals, veterinary clinics with the same city address as the community colleges were called and e-mailed a letter with an anonymous link to the survey.



\section{Survey Validity}

As mentioned above, measures will be taken to ensure the validity of the survey by addressing common issues related to wording of the questions and other common concerns.  One way to help with the instrument validity is to have experts review the survey to make sure there are no issues with the tool before use \citep{ary}. Suggestions for changes from multiple experts were made prior to the study. Two experts were analyzing the survey from the perspective of having seen many descriptive surveys in the past and know how to format and present questionnaires. One expert was analyzing the survey from the perspective of a companion animal nutritionist making sure that all relevant questions were asked.
\par Since this is a descriptive needs assessment study, a descriptive survey of a portion of the target population is very suitable to accomplish the research objectives.  Descriptive surveys are an effective way to gather information on what participants know about a topic that is not currently taught in the formal community college setting.  As descriptive surveys can be vulnerable to respondents lying or choosing an obvious answer or not answering some questions at all, the researchers will account for this in making the survey.  Questions will be worded so no obvious researcher-preferred answer, and questions will be expanded in detail to encourage consistency in responses \citep{ary}.

% Below \subsubsection
% Sectional commands: \paragraph and \subparagraph may also be used

\section{Data Collection}

Qualtrics\textsuperscript{\textregistered} was used to create, distribute, and collect the survey data from the research participants.  Two to three contacts occurred with the research participants and included the following: 1. Phone call: only occurred for some, 2. E-mail with questionnaire link, 3. 2nd E-mail with thank you and questionnaire link.  The phone call was a limited contact to community colleges to acquire instructor e-mail addresses and to veterinary clinics that did not have an e-mail contact on their webpage. The phone call consisted of an introduction of myself and a short description of the survey and it's importance to my research. The e-mail of the clinic was acquired during this phone call, as well, if the clinic agreed to participate. The first e-mail contact described the study in more detail and stated the importance of impacts of the research on community college programs for veterinary technicians, veterinary assistants, and animal scientists. All who were surveyed were informed that the survey was entirely voluntary, and they were not required to respond to the survey.  Information for the IRB at Iowa State University was provided for participant convenience if they have questions about the research. The first e-mail also contained a link to the survey for participants to access Qualtrics\textsuperscript{\textregistered} easily and anonymously. The second e-mail was sent one week after the first e-mail. This was enough time to allow students and professionals to read the e-mail, ask questions of the IRB if they had some, and possibly take the survey already. This second contact still included the link to the survey for participants to access and complete at their convenience. Participants were thanked for their time taking the survey in both e-mails. Due to the way that e-mails had to be sent, particularly for the community college students, there is no real way to calculate a response rate since the number of people with access to the survey link is unknown.

\section{Data Analysis}

Qualtrics\textsuperscript{\textregistered} was used to collect and analyze data gathered from the completed surveys of this study. The main type of data processing will be descriptive statistics, as is applicable for a descriptive survey. To generate the dot plot figures, the statistical program R was used with the data exported from Qualtrics\textsuperscript{\textregistered}\citep{R}.

\section{Limitations}

The major limitation for this study was the accessibility of individuals in the target population. This is due to the privacy policies of the colleges, as well as veterinary clinics only providing one e-mail address for the whole clinic. Because of these limitations, the true number of people with access to the survey is unknown. Therefore, researchers will be limited in reporting an accurate response rate and in making definitive conclusions about the entire population. The researchers do know how many schools and clinics were contacted and can assume a minimum number of employees or students that had access to the questionnaire.


%\begin{table}[h!tb] \centering
%\isucaption{This table shows a standard empty table}
%\label{nothing}

%\vspace{ 2 in}
%\end{table}
%\begin{figure}[h!tb] \centering

%\vspace{ 2 in}
%\isucaption{This table shows a standard empty figure}
%\label{moon}
%\end{figure}