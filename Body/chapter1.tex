% Chapter 1 of the Thesis Template File
\chapter{OVERVIEW}

Humans have had animals as companions for thousands of years, and now many species have been domesticated to live closely with their human caretakers. For example, evidence of the first domestic dog is from about 15,000 years ago \citep{savolainen}, and the domestication of the cat is dated to somewhere between 9,500 and 3,600 years ago \citep{driscoll}.  Although humans have lived with these domestic animals as their pets for so long, the foods available to feed companion animals have changed drastically over the years.
\par The first dog biscuit was inspired in 1860 by “hard tack,” a treat made up of water, salt, and flour that was used as food for sea voyages \citep{tudor}. A salesman named James Pratt from Ohio visited England and saw British sailors feeding the “hard tack” to dogs, so he came up with the idea for the first dog biscuits \citep{tudor}.  Other than these biscuits, dogs and cats were mostly fed table scraps and raw meat until the 1920’s, when dehydrated meals, canned food, and pellets were made from meat and grain mill scraps.  At this point, only those wealthy enough to buy pet food would purchase these new meals, so many household pets still ate table scraps supplemented with whatever they could hunt or forage outside \citep{tudor}.  During World War II, food rationing made table scraps rare, and people that could afford pet food began mostly buying dry pet foods.  After the war was over, there was a major economic boom that allowed more people to start buying pet food \citep{tudor}.  In the 1950’s, a major pet food company discovered how to make dry food pieces of any shape by creating a soup of ingredients and heat processing so that it “popped” into light weight pieces of food, now called kibble \citep{tudor}.


\section{Introduction}

Currently, there are hundreds of pet food brands and formulas to choose from in the market.  This industry started just over 60 years ago with a vast majority of the growth in the pet food industry occurring in only the last 30 years’ time to the point where the industry overall generates over 20 billion dollars every year since 2013 \citep{tudor,prweb}.  In 1958, as the industry grew and more chemicals were being used to process foods, the FDA modified the existing act that regulates the pet food industry to make sure additives were approved for use in foods \citep{fda}.  Other significant regulations have been put into place since then, including regulations at the state level \citep{fda,hillestad}.  Another area of growth that changed with the growth of pet food companies is the pet supplies stores that did not previously exist.  Petco started as UPCO, a veterinary supplies delivery service, in 1965 and became Petco in 1979, expanding outside of California in 1980 \citep{petco}. The first stores that lead to PetSmart opened in 1987 in Arizona as PetFood Warehouse stores \citep{petsmart}. For the most part, society has changed with the times to regulate pet foods and grow the market and overall economy in related areas.  One area that has not been adapting appropriately is education related to the pet food industry.
\par Though there is a need for nutritional education, nutrition is already recognized as a specialty that veterinary technicians can acquire. This shows that nutritional knowledge is considered important in the industry, but the requirements for a veterinary technician to become certified in nutrition are largely focused on clinical nutrition practices and general nutrition\citep{avnt}. There is much more to companion animal nutrition than what currently and mainly fulfills the requirements for this specialty, but this is where reliable education is difficult to find.
\par Most education on pet foods occurs through the internet, which can be helpful to learners in some situations.  Websites like www.peteducation.com and www.dogfoodadvisor.com are just two popular sources among many that pet owners may find information on pet foods and their ingredients.  However, like many other things on the internet, there are many differing opinions on pet food quality and ingredients.  Many sources and people in the pet food industry disagree about what is beneficial or not beneficial in pet foods or which pet foods are so-called “good” or “bad” and what makes them “good” or “bad.”  Just from reading pet food company websites, mixed messages are sent depending on what each company is selling.  Some advertise the benefits of whole grain ingredients and bragging about lacking certain grains labeled as “fillers \citep{blue}.”  Other companies claim that the grains sometimes called “fillers” such as corn, wheat, and soy are actually beneficial to pets in the long run \citep{purina}.  None of these sources of information can be considered fully reliable, and if only one or two sources are used, some valuable information may not be included in the education received.


\subsection{Statement of the Problem}

Is there a need for additional companion animal nutrition education at the community college level for animal science, veterinary technician, and veterinary assistant students?

\subsection{Objectives of the Study}

A descriptive study is appropriate to assess the need for an educational program on pet food brands, formulas, composition, and ingredients directed toward veterinary technician, veterinary assistant, and animal science students in community colleges.  This study has the following objectives:
\begin{itemize}
    \item Identify the demographics of the students in veterinary technician, veterinary assistant and animal science programs in Iowa community colleges.
    \item Identify the relevant demographics of practicing veterinary technicians and veterinary assistants in clinics in the same town or city as Iowa community colleges with the veterinary technician, veterinary assistant, and animal science programs.
    \item Identify differences in opinion or knowledge level based on position, student status, age, or pet ownership.
    \item Identify the knowledge level of veterinary technician, veterinary assistant, and animal science students.
    \item Identify the level of interest of participants in an educational program on pet foods.
    \item Identify perceptions regarding differing diet philosophies (e.g. home-cooked, kibble, raw, etc.)
    \item Identify the knowledge level of practicing veterinary technicians and veterinary assistants in the areas around each community college.
    \item Describe the perceived value of unbiased carnivorous companion animal nutrition education.
\end{itemize}

\subsection{Need for Study}
A needs assessment cannot guarantee the success of a program, but carefully utilizing needs assessments can significantly reduce the uncertainty of program acceptance and vitality\citep{queeney}. Needs assessment is a process used to identify gaps and discrepancies between what currently is and what should be\citep{queeney}. In the context of this study, needs assessment is necessary for identifying discrepancies in the carnivorous companion animal nutrition education of veterinary technicians and assistants. Without considering what the educational needs of this population are, educators risk creating the wrong programs at the wrong times in the wrong locations for the wrong population\citep{queeney}. Hopefully, and ideally, this study is just the first in a series of needs assessments that become the building blocks of a long-term strategy for a program development process\citep{queeney}.

\subsection{Definition of Terms}
\begin{itemize}
    \item Veterinary technician - ``A veterinary technician [is] a person who has graduated from a veterinary technology program accredited or approved by the Committee on Veterinary Technician Education and Activities (CVTEA) of the American Veterinary Medical Association, or other accrediting agency approved by the Board, or a person who has received equivalent training as set forth in the rules of the Board." \citep{idaho}
    \item Veterinary assistant - ``They support animal doctors and technicians in their daily tasks by cleaning and maintaining equipment, feeding, exercising and grooming patients, preparing and sanitizing surgery suites, restraining and handling patients, and clerical and administrative work."\citep{allied}
    \item Needs assessment - ``A decision-making tool for continuing educators' use in identifying the educational activities or programs they should offer to best meet their clients'--and society's--educationals needs."\citep{queeney}
    \item Clinical nutrition - ``Clinical nutrition is the study of the relationship between food and a healthy body. More specifically, it is the science of nutrients and how they are digested, absorbed, transported, metabolized, stored, and eliminated by the body."\citep{umm}
    \item RACE - Registry of Approved Continuing Education -- ``one of the four key programs provided by the [American Association of Veterinary State Boards]. The purpose of the RACE program is to develop and apply uniform standards related to providers and programs of continuing education (CE) in veterinary medicine. Our goal is to serve and support the [American Association of Veterinary State Boards] Member Boards by ensuring that all RACE-approved programs meet appropriate standards of quality continuing education."\citep{race}
    \item NAVTA - National Association of Veterinary Technicians in America -- ``formed in East Lansing, Mich., in 1981 with the goal of allowing veterinary technicians to give input on national issues involving the veterinary profession...Since then, NAVTA has grown and seen many successes, including the declaration of National Veterinary Technician Week, the formation of the Committee on Veterinary Technician Specialties, the development of a scholarship program, and much more."\citep{navta}
    \item AVMA - American Veterinary Medical Association -- ``a not-for-profit association representing more than 89,000 veterinarians working in private and corporate practice, government, industry, academia, and uniformed services. Structured to work for its members, the AVMA acts as a collective voice for its membership and for the profession."\citep{avma2}
    \item ASPCA - American Society for the Prevention of Cruelty to Animals -- ``a national leader in the areas of rescue, adoption and welfare and has worked tirelessly for over 150 years to put an end to animal abuse and neglect."\citep{aspca}
\end{itemize}

% Below \subsubsection
% Sectional commands: \paragraph and \subparagraph may also be used


