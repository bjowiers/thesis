% Chapter 2 of the Thesis Template File
%   which includes bibliographic references.
\chapter{LITERATURE REVIEW}



\section{Introduction}
This chapter provides an overview of available sources of companion animal nutrition information that veterinary technicians and assistants can access and a framework for a descriptive study. This chapter also presents the current companion animal nutrition information available, shortage of companion animal nutrition information available, and research needs in companion animal nutrition education.



\section{Descriptive Research}
There are a few studies published related to educating people about carnivorous companion animal nutrition. Most of the studies discuss the education of dog and cat owners, but not veterinary staff. One such study had a component about educating the owners of obese pets and provided monthly education to pet owners about nutrition and other factors affecting weight\citep{yaissle}. The study found that educating owners on a regular basis did not affect the amount of weight lost compared to the control group that did not have the education every month\citep{yaissle}. A second study on obese pets used client educational sessions on a weekly basis but was not able to definitively show whether it was education or exercise that improved weight loss rates\citep{chauvet}. A study in Australia focused on educating communities about general dog health where poor dog health was directly related to poor human health\citep{constable}. \cite{constable} found that identifying the needs local to each community was essential for implementing a dog health program that effectively met the needs of each community. Once people realized that relevant information was available to them as opposed to generalized information not all people needed, the communities called for more education\citep{constable}. Another article discusses the importance of veterinarians using evidence-based medicine in clinical nutrition for animals with diseases or to prevent disease such as periodontal disease\citep{roud2}. \cite{roud2} suggests that practitioners in the veterinary field should understand risks and benefits of different nutritional options and be able to counsel owners on nutritional care in clinical situations in particular.
\par Though this research is important and relevant to this study, none of it is very similar to this study. Two studies worked on educating pet owners about pet obesity and factors that affect weight, but both were about educating owners, not veterinary clinic staff, about a specific part of nutrition\citep{yaissle,chauvet}. Neither \cite{yaissle}, nor \cite{chauvet} did a needs assessment for the groups they were attempting to educate. \cite{constable} was similar in that a needs assessment was performed in four separate communities to identify what people wanted to learn. Additional research was done on what afflicted dogs in the communities to see what residents might not know they should learn about\citep{constable}. However, \citep{constable} only researched the needs of and educated the communities, not the veterinary technicians and assistants. \cite{roud2} presents cases in which clinical nutrition knowledge is important for practitioners to have and explicitly states the importance of possessing such knowledge, but only proposes the use of evidence-based medicine in determining the effectiveness related to clinical nutrition. 

\section{Current Carnivorous Companion Animal Nutrition Education Availability}
Many sources for information on carnivorous companion animal nutrition exist today, but there is a lot of discrepancy on accurate and unbiased information. Every pet food company will have a philosophy on cat and dog nutrition, and their diets reflect their philosophies. \cite{blue}, for example, comes forward to say that certain ingredients will never be found in Blue Buffalo diets due to common pet allergies. On the other hand, \cite{purina} claims that these same ingredients are rarely problems for pets and actually have benefits nutritionally, emphasizing the importance of studying effects of a complete diet rather than individual ingredients. Companies with these defined philosophies will rarely or never suggest that homemade diets are best for pets, but \cite{strombeck} would disagree with these pet food giants. In fact, \cite{strombeck} states that veterinarians should do more to counsel pet owners on nutrition for pets than simply recommend a commercial pet food. In his book, \cite{strombeck} says that pets can be fed foods consumed by humans and when done correctly can prevent medical problems. Another source of nutrition information that is specifically directed toward veterinary technicians is the NAVTA with options for completing RACE credits\citep{ce}. An article in one issue of NAVTA's journal publication discussed the importance of veterinary technicians sharing nutritional knowledge with pet owners since pet owners may not acquire the knowledge another way\citep{burns}.  \cite{burns} presented situations of potential challenges for veterinary technicians to overcome, but the article specifically said to tell clients not to feed pets people food due to risk of obesity and resistance to eating a balanced diet. \cite{burns} did, however, offer options of vegetables as treats instead of only commercial pet treats. In addition, the authors are associated with either Nestle Purina Pet Care or Hill's Pet Nutrition\citep{burns}. Association with a pet food company suggests that there may be some bias for the authors to recommend commercial pet foods over a home-prepared diet that other sources would promote instead. All of these sources of companion animal nutrition information can be useful in some way, but all are also likely to be biased against the others for a number of reasons. 

\par There are endless sources of information online and in print that all have differing opinions and research to back up claims. There are also sources that are more reliable and less biased than the aforementioned sources, such as nutrition classes at colleges. The community colleges in Iowa that have programs for veterinary technicians and assistants all have a nutrition course that is required for the completion of the certification. Des Moines Area Community College actually has two nutrition classes offered, with one that focuses on dogs and cats and the other that focuses on livestock\citep{dmacc}. Muscatine Community College, Iowa Lakes Community College, Iowa Western Community College, and Kirkwood Community College also offer small animal nutrition courses for students to take toward the completion of veterinary technician or assistant degrees or certifications\citep{mcc,iwcc,ilcc,kcc}. Hawkeye Community College and Western Iowa Tech Community College have general nutrition courses, but none specific for companion animals\citep{hcc,witcc}. These college courses designed purely to educate the learners are some of the best sources of carnivorous companion animal nutrition information available, but one course on the basics may not be enough.

\section{Shortage of Carnivorous Companion Animal Nutrition Education Available}
Even though there are many sources of information that veterinary technicians can consult, there is still a shortage of education in the field of companion animal nutrition. Two of Iowa's community colleges do not have the option of a small animal or companion animal nutrition course that would focus on cats and dogs. One community college in the state, Northeast Iowa Community College, does not offer any nutrition course for their veterinary assistant and large animal veterinary technician program of study\citep{nicc}. Course descriptions for the small animal and companion animal nutrition courses offered at Iowa's community colleges all say something similar to the following: ``Covers essential nutrients and the roles of each in an animal's metabolism, with an emphasis on the nutritional management of dogs and cats. Basic clinical and therapeutic nutrition are covered in depth. Includes analysis of many commercial pet foods."\citep{kcc} The course content in these existing community college courses is very useful in the education of veterinary technicians and assistants. These courses introduce students to the basics of cat and dog nutritional needs and go over very important aspects of nutrition related to the health of pets. One thing that none of the course descriptions mention is home-prepared diets or raw diets, which are both being used by many pet owners today\citep{oconnor}. Also, all of the small and companion animal nutrition courses mention analyzing pet foods, but only specifically analyze commercial pet foods\citep{kcc}. Additionally, although many commercially available pet foods are likely discussed in each of the courses, veterinary technicians may still possess very little knowledge on many of the pet food brands that are easily available to the average pet owner. Due to the size of the pet food industry, it would be out of the question to expect veterinary technicians or assistants to know details on every brand available from every store. What would be reasonable is to expect veterinary technicians to know at least basic details about leading brands as well as brands tailoring to niche markets such as freeze-dried foods, commercially sold refrigerated or frozen raw options, and the high protein, meat dense specialty brands. Unbiased college courses or RACE credit options through NAVTA or other sources do not currently offer education much if at all on these specific topics.

\section{Need for Research in Companion Animal Nutrition Education}
Most pet owners, today, are feeding commercial pet food products to their pets, but most pet owners also know fairly little about pet nutrition\citep{oconnor}. Many pet owners turn to their veterinarians for recommendations, but if that recommendation does not work out for the animal or its owner, the owner often turns to online sources for help. Due to the nature of searching online for pet nutrition suggestions, the average pet owner is more likely to come across the option of home-prepared or raw pet food formulation now compared to in the past. Both the ASPCA and AVMA recommend against raw diets for pets because of the risk of bacterial contamination that would result in a sick pet\citep{oconnor}. Though this is a reasonable recommendation, many pet owners are willing to take the risk, regardless of their veterinarian's opinion. Differing opinions between client and veterinarian could result in friction in the relationship, potentially causing the pet owner to switch veterinarians. Pet owners will also find pet foods that the staff at their veterinary clinic have never heard of because the industry is large enough that veterinary staff are not likely to know nearly every brand of commercial pet food. This is unfortunate for veterinary clinics, whose staff then end up losing some credibility with pet owners. The veterinary sources for the average pet owner should be much more dependable than a website when a pet owner has a question in pet nutrition. Currently, there is not enough reliable, unbiased information available for veterinary technicians and assistants to get accurate ideas of different pet diet options. Because of this deficit in the industry, it is likely that veterinarians, veterinary technicians, and veterinary assistants also refer to websites for assistance at times. This can also be a problem for veterinary clinics if the technicians, assistants, or veterinarians, themselves, develop strong opinions of pet nutrition based on a few websites that may be biased or subscribe to one theory of pet nutrition.
\par The veterinarian oath states, ``Being admitted to the profession of veterinary medicine, I solemnly swear to use my scientific knowledge and skills for the benefit of society through the protection of animal health and welfare, the prevention and relief of animal suffering, the conservation of animal resources, the promotion of public health, and the advancement of medical knowledge. I will practice my profession conscientiously, with dignity, and in keeping with the principles of veterinary medical ethics. I accept as a lifelong obligation the continual improvement of my professional knowledge and competence."\citep{avma} This oath emphasizes many aspects related to animal health, but does not specify a nutritional philosophy that should be adhered to by every practice. Instead, practices need the knowledge and education to make wise decisions specific to each animal and client, including if a client uses a diet that the veterinary staff disagree with. Ideally, veterinary technicians and assistants would have the knowledge to be able to assist with home-prepared diet recipes and also know enough about commercial pet foods to make recommendations if the pet owner wants something different than what the clinic sells. This is where much of the need for research on carnivorous companion animal nutrition lies: application of practical knowledge on diets for healthy animals, as well as ill animals.

% Below \subsubsection
% Sectional commands: \paragraph and \subparagraph may also be used

\section{Conceptual Framework}
Needs assessment is performed to help determine program content, format, delivery mode, audience, and marketing concerns such as how to inform the audience of the program\citep{queeney}. At the most basic level, a needs assessment is used to determine what programs are necessary and what content is most needed. Conducting a needs assessment helps educators plan a program or series of programs that will be more likely to meet both the goals of the educators and the goals of the potential audience. Not all needs assessments have to be large projects or require many resources, and one option for a small-scale needs assessment is a survey. Surveys allow for contacting an audience that may be dispersed geographically or unable to attend for other forms of needs assessment\citep{queeney}. In particular, a questionnaire allows for a low cost option of contacting a dispersed population. Careful planning is required of the researchers conducting the needs assessment related to all the parts of survey design, such as population identification, question design, and survey format\citep{queeney}.
\par In the development of questions, it is necessary to scrutinize the wording to ensure that ``value-laden" phrases that could influence participants are not present\citep{queeney}. Also, wording must be clarified to avoid misinterpretation or prejudice when respondents see the questions. In addition, questions that would not be useful to the researchers in some way should be excluded, even if interesting to the researchers. As for type of question, close-ended or open-ended questions can be used in a survey, however, if possible, close-ended questions are preferable\citep{queeney}. \cite{queeney} set forth some basic guidelines for writing survey questions as follows:
\begin{enumerate}
    \item ``Be succinct."\citep{queeney} Often, respondents won't want to read long questions due to the thought the answer will take more time, as well. There is also higher risk of misunderstanding with drawn out questions. 
    \item ``Set a positive tone."\citep{queeney} A positive, respectful tone to a survey can display a constructive attitude to participants while remaining neutral with respect to content of the survey.
    \item ``Use proper language and grammar."\citep{queeney} Unprofessional language such as slang or jargon, as well as grammatical errors, should not be present in the survey because simple sentence structure will best facilitate comprehension.
    \item ``Request one piece of information at a time."\citep{queeney} Separating the items of interest is the only way to ensure clarity of the results so that data is obvious for each item in the survey.
    \item ``Use branching questions when bifurcation is necessary."\citep{queeney} This type of question allows participants to only answer relevant questions to themselves, taking into account their previous answers.
    \item ``Offer clear response options."\citep{queeney} In close-ended questions, clarity of options and type of response, such as choosing a single answer or multiple answers, is very important to participant comprehension\citep{queeney}.
\end{enumerate}
\cite{queeney} also has some considerations specific to questionnaires that are important for this type of needs assessment. The questionnaire-specific considerations are as follows:
\begin{enumerate}
    \item ``Format."\citep{queeney} Booklet format is discussed over stapled pages, but this study performed an online needs assessment questionnaire, instead.
    \item ``Introduction."\citep{queeney} This is at the very beginning of the questionnaire and briefly describes the purpose of the questionnaire in one or two sentences.
    \item ``General Instructions."\citep{queeney} Basic, clear, concise directions are stated next and are worded politely.
    \item ``Transitional Statements."\citep{queeney} Transitional statements can be used to help separate sections of a longer questionnaire.
    \item ``Conclusion."\citep{queeney} The end of the questionnaire includes a statement thanking participants for their responses in the questionnaire\citep{queeney}.
\end{enumerate}
\par This study follows these suggestions as closely as reasonable for an online questionnaire, which is different in some ways from the written format that \cite{queeney} discusses more specifically. This study set out to identify programs and content needed for veterinary technician, veterinary assistant, and animal science students using a needs assessment in the form of an online questionnaire. The needs of this population were not assessed over a period of time for this study, as would be considered optimal in many situations\citep{queeney}. This study is assessing the educational needs in the area of carnivorous companion animal nutrition despite the existing courses and other sources of information. An evaluation of additional needs in this educational area is necessary to ensure that veterinary technicians and assistants graduating from Iowa's community colleges have the nutritional knowledge necessary to work in Iowa's veterinary clinics.
\section{Summary}
This chapter provided a basic overview of the current research in companion animal nutrition education. The brunt of research found in the subject is related to educating pet owners or entire communities about nutrition related to obesity or overall health\citep{yaissle,chauvet,constable}. One study discussed the importance of applicable knowledge specifically in clinical nutrition, but did not propose an educational program for acquiring this knowledge\citep{roud2}.
\par Next, this chapter described current informational sources in companion animal nutrition. Some sources are reliable, such as community college courses on the basics of companion animal nutrition, but other sources are much more biased. Most of the sources mentioned have some level of bias against other types of diets and nutritional philosophies. Pet food company websites cannot be trusted for information because every pet food company has a different philosophy from the next company. Additionally, all pet food companies are trying to attract customers in order to profit, so their philosophies and diets will be made to appeal to the customer base to get a sale. Some web sources are a bit less likely to be biased, but can still result in strong opinions about ingredients or type of diet that should or shouldn't be fed. Overall, comparatively few sources of nutritional information are reliable and purely educational in nature.
\par Third, the current chapter discussed the shortage of information available to veterinary technician, veterinary assistant, and animal science students. Though a class is available at most of the community colleges in Iowa, not all have that option. There is also a large deficit in the education of pet food brands available outside of the brands commonly sold in veterinary clinics. Additionally, there is incredibly little information reliably available to this group related to raw, home-prepared, and freeze-dried foods that are rapidly increasing in popularity\citep{oconnor}. Part of this shortage could be related to the AVMA recommending against using these types of diets due to supposed likelihood of obesity in the case of home-prepared with people food\citep{burns}, as well as the chances of contracting illness from contaminated food products\citep{oconnor}. This shortage could be detrimental to veterinarian-client relationships in the future considering the significant growth in raw pet food product sales\citep{oconnor}.
\par Then, the need for research in the area of companion animal nutrition education was outlined. Based on the shortage information, it is clear that the community colleges that do not currently have a small animal nutrition class could likely benefit from adding one. Also, an additional course could likely be created for educating interested students in the areas of pet food brands, common pet food formulations, commercial raw pet foods that are refrigerated or frozen, commercial raw freeze-dried products, home-prepared cooked diets, home-prepared raw diets, and possibly more. Fulfilling these needs could result in educating pet owners who feed raw diets on safe practices to make illness less likely for the humans and their pets.
\par Lastly, the framework for this study was discussed as a needs assessment using a survey in the form of an online questionnaire. Queeney's (1995) guidelines for writing survey questions and outline for writing a questionnaire were helpful in determining whether the questionnaire would yield accurate, usable results. Because these guidelines were adhered to, the instrument was more successful in determining educational needs of this study's population.