% Chapter 5 from the standard thesis template
%   with a full page figure and a sideways table.
\chapter{DISCUSSION AND SUMMARY}

\section{Introduction}
This chapter focuses on the conclusions based on the results of the questionnaire. The implications for the industry and community colleges are discussed along with recommendations for future study. Discussion of what researchers would do differently if this survey was to be conducted again is also included in this chapter.

\section{Reflections and Discussion}
The demographics of respondents showed that most respondents attend or attended Muscatine Community College. Also, the largest age group was 18 to 25 with a total of 7 respondents in that range. Based on the question of position, 3 participants were veterinary technician students, but based on submission date, 10 respondents were students. Based on the question of position, 5 respondents were veterinary technicians, and 1 was a  veterinary assistant. All but one respondent owns at least one dog or cat, with most respondents feeding kibble and two feeding a home-prepared diet. Based on the total amount of demographic data, shown in Table~\ref{tab:my_label}, the responding group is fairly diverse in terms of educational background, number of pets, current position or student status, and age.
\par Due to the very small response number, researchers cannot make absolute conclusions from the data of this study, however, even with the low number of participants, trends can be seen in the data. Based on the results as a whole, veterinary technicians and assistants seem to be getting into their jobs with a decent amount of basic nutrition knowledge specific to cats and dogs. This was not known before and should be researched further to confirm the findings with a larger sample. Also basic nutrition knowledge is acceptable for this group of respondents, there are some obvious gaps in the nutrition education currently received at the community college level. Currently, courses are focused on education of the function of the digestive system and general information about commercial pet foods and pet food labeling. These areas are important for students to be knowledgeable in, however, because of the direction that the pet food industry and pet owners are headed, students may need more than this basic education. This idea is supported by the knowledge ratings in specific topics, such as freeze-dried foods, raw diets, and home-cooked diets were all lower than the basic knowledge areas like digestive physiology and kibble diet advantages and disadvantages. Also, the familiarity levels of different pet food brands was very telling in that for both students and veterinary clinic staff, many up and coming premium pet foods were rated as not familiar. These premium pet food brands are available at specialty pet food stores, and some are available at Petco, PetSmart, or both. Researchers believe this shows a need for an educational program centered on these gaps in nutrition education because veterinary clinic staff, current and future, will get many questions related to pet diets. Veterinary technicians should be prepared for this by knowing what is available to average consumers and what owners see in other diet compositions besides kibble.
\par Participants indicated they are very interested in an educational program on carnivorous companion animal nutrition. This is encouraging to researchers considering the noted gaps in current knowledge held by participants. The interest in learning more is important considering the value and importance of a strong nutritional background are also highly rated among participants. This indicates that if there was a program, there would be interest and some added value after the program conclusion.
\par Some interesting observations of the opinions of different diet options for pets can also be made. For example, a significant majority of the respondents have a positive opinion of kibble, which was somewhat expected by researchers. However, one respondent had a negative opinion of kibble and believes all pets should be fed "fresh" diets. This was much less expected by researchers due to the majority of the people in the industry surveyed have no problem with kibble. The opinions on home-prepared diets were more positive than researchers anticipated, but this is a good thing for the industry considering the growing market for raw and home-cooked diets. There are still many negative opinions about these diet formulations, mostly centered around the risk of contracting food-borne illness or malnutrition due to an unbalanced diet. Researchers believe these opinions are valid, but that the perceptions of the diets overall need to change so that pet owners can openly discuss how to formulate a diet to be balanced and safe. Some participants were neutral on the home-prepared diets, which would be preferable for typical veterinary clinics as this would more likely result in an open mind to the option if a client wanted to discuss the viability of a raw or home-cooked diet. Unlike with kibble and home-prepared diets, no respondents responded with a positive opinion of commercially made grain free diets. Most were neutral, acknowledging that it works for some pets, but some expressed negative opinions because of the opinion that grains are good for pets. This was a surprise for researchers because grain free foods are becoming very common to the point that grain free diets are now available at stores, such as Wal-Mart. The opinions of the different diet types did not differ across groups for age, position, student status, or any other. It is possible there wasn't enough data to see the trend for these opinions. Researchers anticipated that among respondents, at least one or more would have seen enough positive results with grain free foods to have a positive opinion of them in general. Overall, the opinions of these diets show that an educational program may be capable of helping change opinions of diet options and help veterinary technicians with methods of recommendation without creating a rift between them and the client. Making recommendations to pet owners can involve compromise and veterinary clinics can be more involved in diet choices for pets if this occurs.

\section{Recommendations}
Based on the data collected in the current study, researchers suggest considering an educational program be designed and offered at community colleges for veterinary technicians students to take optionally. Another option would be to create a new program for continuing education to offer veterinary technicians currently in the workforce since there would be interest and veterinary technicians need RACE credits on a regular basis to continue their certification. These recommendations are best-case-scenario suggestions making the assumption that the current sample of respondents is representative of the population. If the results of this study are indicative of the entire population, researchers would recommend that independent carnivorous companion animal nutrition experts be consulted for designing and conducting unbiased educational programs for interested parties. Specifically, the programs should include the topics of home-prepared raw diet formulation, home-prepared cooked diet formulation, commercially manufactured and sold raw diets(freeze-dried, refrigerated, and frozen), additional commercial refrigerated and frozen diets and treats, pet food brands available to average customers, and what those pet food brands offer. This program could be a condensed seminar format, but could also be expanded to a semester-long course.
\par Another option that researchers would recommend as an option would be an extension program at the county level. This would make the program available in areas outside the community college communities so that more veterinary technicians could have access to the program. This option was not specifically explored in this study, however, so the recommendation is tentative, assuming that the interest in the program would not depend on the community college being the source. It is possible that community colleges that already have curriculum approval for certified veterinary technician degrees may be more likely to get a continuing education course approved for RACE credits. A continuing education course offered through a community college with a veterinary technician program would be able to assist veterinary technicians acquire the credits needed to become specialized in nutrition\citep{avnt}. The possibility of veterinary technicians using this program to become specialized would help to attract participants the a continuing education program. This is something to be researched and considered in any program planning.
\par A third option researchers would recommend is that community colleges consider setting up a webpage with information with factual, unbiased, all-inclusive information on carnivorous companion animal nutrition that is accessible to veterinary technician students and veterinary technicians. This could possibly allow veterinary technicians and assistants to acquire their RACE credits if there were interactive modules on the website that allowed the learners to show their progress in the topics.
\par Regardless of the audience the course would target or who would offer the program, the curriculum should be the same for each situation. As previously stated, the topics ought to include home-prepared raw diet formulation, home-prepared cooked diet formulation, commercially sold raw diets, refrigerated and frozen foods and treats, and pet food brands and what they offer. The goal of the program would be to equip program participants with the skills and knowledge to fully, accurately, and objectively assist clients with diet choices for the clients' pets. To achieve this, program content should be discovered by learners through activities supervised by the instructor. The learners would be allowed to develop their own opinions and critically analyze each topic in the course to cement nutritional problem-solving skills for use in practice. By the end of the course, program participants should be fully capable of developing home-prepared diets, cooked or raw, for both cats and dogs. Participants should be well-versed on commercially available products and the commonly available pet food brands to more effectively evaluate pet health and nutrition. Learners will also know the benefits proven or advertised for different types of diets and advise clients on validity of the claims. Lastly, program participants would be knowledgeable enough in these areas and how they relate to clinical nutrition in order to make alternative diet programs for clients unable or unwilling to buy a prescription diet for certain issues affected by nutrition. If the program is designed with the assistance of nutrition experts and experts in education, all of the goals in this model curriculum can be achieved.
\par Because researchers cannot be at all certain this sample actually represents the entire population in Iowa, further research is strongly recommended. A survey similar to this, but perhaps more focused on educational gaps identified in this study, could be conducted on a larger sample of Iowa's veterinary technicians and veterinary technician students. Though this sample is very small, there is an obvious trend showing a desire to learn more in the areas of pet nutrition. Researchers believe this trend would be seen in a larger sample, as well, supporting the idea that the entire population would be interested in learning more about carnivorous companion animal nutrition.


\section{Suggestions for Improvement}
A number of things would be done differently if researchers were to conduct this survey, or another like it, again. First and foremost, researchers would find ways to know how many people have access to the survey so that there is a response rate statistic. Unfortunately, the lack of response rate for this study ensures the very limited impact of the research performed. Perhaps, future researchers could build more rapport with potential respondents so they are more likely to respond and researchers can acquire individual e-mail addresses for sending electronic surveys. Another option would be to randomly select a participant to win a prize, such as a gift card, to incentivize people in the sample population to participate. Researchers had considered changing the study proposal with the Institutional Review Board to do just that, but ended up deciding to attempt the study without that incentive. If the researchers were to do a similar study, an incentive would be a part of it. Also, a more minor, but important change would be to the survey, itself. Due to the way that the questions directed respondents, some respondents never saw a question that was relevant to them. The question was the one on position and student status where seven respondents did not answer. The lack of response for most of all of them was because they never saw the question because they answered "no" to the question of whether they worked in a veterinary clinic. If the survey was conducted again, the "no" answers would be directed to a question about student status to make sure the data were accurate without taking into account who had access at what point in the survey. With these improvements implemented, the results and outcome would likely be much more accurate, conclusive, and helpful in determining need of an educational program.

\section{Summary}
 Overall, the area of pet nutrition has been evolving since the very first dog treat in 1860 and the first pet foods manufactured in the 1920's\citep{tudor}. Government regulations on pet food have also evolved at the federal and state levels\citep{fda,hillestad}. Because of this evolution in pet food, the pet industry as a whole has had to evolve, as well. Community colleges are now offering a basic nutrition course specifically for companion animals to veterinary technicians and assistants\citep{kcc}. This course, offered at a number of Iowa community colleges, is definitely better than no small animal nutrition course, as evidenced in the moderate to above moderate knowledge levels of participants in many areas related to the subject. Respondents desire to learn more in the area of companion animal nutrition, and should have access to unbiased nutritional education so as to fairly represent all options available. Anyone participating in an educational program based on the recommendations in this study should feel free to develop his or her own opinion without the influence of the program director or instructor's opinions. Further study is certainly needed to confirm the observations of the current study, but this study is a functional stepping stone for research in the area. The suggested unbiased education in the areas discussed would be extremely beneficial in the industry today and propel veterinary technicians and assistants to the forefront of nutritional decisions of pet owners.