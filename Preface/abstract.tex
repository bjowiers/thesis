\specialchapt{ABSTRACT}

The pet food industry continues to grow and expand with every passing year. Accommodations are made throughout society to address the growing market and customer base, but the education of veterinary professionals has not adequately followed suit. Too many biased sources of nutritional education cause veterinary clinic staff to be underprepared for helping pet owners with nutritional questions or needs. Veterinary clinic staff are knowledgeable in particular areas of nutrition, such as clinical nutrition and digestive anatomy and physiology as well as others, but this is not enough for the current industry growth. New pet food brands and types of food and treats show up in stores every year. Unfortunately, there is presently no course in Iowa designed to educate veterinary technicians and assistants, present or future, on the pet food industry and what is available to pet owners. This study assessed the need for such a nutritional program in Iowa through a questionnaire of veterinary technician, veterinary assistant, and animal science students, as well as veterinary clinic staff. Though the sample size was very small, the data is fairly convincing that there is a need for an educational program, as well as further study in the subject.